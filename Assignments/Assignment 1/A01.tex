\documentclass[12pt]{article}
\usepackage{parskip}
\usepackage[hidelinks]{hyperref}
\usepackage[legalpaper, margin=1in]{geometry}
\begin{document}
    \begin{center}
        \textbf{\Large{ELECENG 2FL3 ASSIGNMENT 1}} \\
        \vspace{2mm}
        \large{Raeed Hassan} \\
        McMaster University
    \end{center}
\hrulefill

\textbf{Application 0: TV broadcast} \\
TV broadcasting is the method of delivering audio and video to a television for consumption. Television is generally broadcasted using terrestrial radio waves (over-the-air television), transmitted through coaxial cables (cable television) or relayed from satelittes (direct broadcast satelitte) \cite{1.1}. In Canada, digital television uses the 700 MHz band, using channels 52-69 which occupy 6 MHz frequency ranges from 698 MHz to 806 MHz \cite{1.2}.

\textbf{Application 4: Satellite radio} \\
Satellite radio is a  broadcasting-satelitte service that uses satellites to transmit radio waves to deliver audio that is received by the public, typically used by motor vehicule occupants \cite{2.1}. The frequencies allocated for broadcasting-satelitte services in Canada fall into the 2300 MHz - 24500 MHz frequency range, which is in the microwave range \cite{1.2}.

\textbf{Application 20: Fiber-optic networks} \\ 
Fiber-optics networks are networks of that share and transmit information through fiber-optic communication. Fiber-optic communication is a method of transmitting information through the process of sending light through fibers of glass, and is used for a wide variety of applications, most notably for internet access \cite{3.1}. The fibers used for fiber-optic communication can operate at wavelengths ranging from 800 nm - 1600 nm, however the most common wavelengths used in fiber-optics are 850 nm (353 THz), 1300 nm (231 THz), and 1550 nm (193 THz) \cite{3.2}. These frequencies are in the infrared range.

\medskip

\begin{thebibliography}{10}
    \bibitem{1.1}
    M. J. Fisher, D. E. Fisher, A. M. Noll, D. G. Fink, "Television," in \textit{Encyclopedia Britannica}. Encyclopedia Britannica, [Online document], Feb. 2016. Available: \url{https://www.britannica.com/technology/television-technology}. [Accessed Jan. 19, 2020]. 

    \bibitem{1.2}
    Innovation, Science and Economic Development Canada, "Canadian Table of Frequency Allocations," \textit{Innovation, Science and Economic Development Canada}, Apr. 2018. [Online]. Available: \url{https://www.ic.gc.ca/eic/site/smt-gst.nsf/eng/sf10759.html}. [Accessed Jan. 19, 2020].

    \bibitem{2.1}
    The International Telecommunication Union, \textit{The Radio Regulations}, 2016 edition, The International Telecommunication Union, 2016.
    
    \bibitem{3.1}
    The Fiber Optic Assocation, "Topic: Fiber Optics: Basic Overview," \textit{The Fiber Optic Assocation}, 2019. [Online]. Available: \url{https://www.thefoa.org/tech/ref/basic/basics.html}. [Accessed Jan. 19, 2020].

    \bibitem{3.2}
    The Fiber Optic Assocation, "Understanding Wavelengths In Fiber Optics," \textit{The Fiber Optic Assocation}, 2019. [Online]. Available: \url{https://www.thefoa.org/tech/wavelength.htm}. [Accessed Jan. 19, 2020].
\end{thebibliography}
\end{document}